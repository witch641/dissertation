\chapter{Guideline}\label{chap:guide}

In order to facilitate the use and better demonstrate the excellent features of LaTeX typesetting, the framework and file system of NEU-Thesis has been carefully handled, modularizing and encapsulating each function and board as much as possible. Moreover, after getting to know and understand LaTeX, you will find its attractive features compared to Word-based typesetting systems. So, if you are a beginner, please do not retreat, try a little and persevere in order to appreciate the extraordinary charm of LaTeX, and improve your knowledge of its use by reading related materials such as LaTeX Wikibook\cite {wikibook2014latex}.

\section{Have a try}

\begin{enumerate}
    \item Install the software: \textbf{\href{https://www.overleaf.com/}{Overleaf} is highly recommended.} Install the LaTeX compilation environment according to the operating system used and the information in the chapter ~`\ref{sec:system}'.
    \item Get the template: Download the \href{https://github.com/sci-m-wang/NEU-Thesis-en}{NEU-Thesis} template and unzip it. The NEU-Thesis template provides not only the corresponding class files, but also all the elements to complete the dissertation, including references, etc. Therefore. When downloading, it is recommended to download the whole NEU-Thesis folder instead of the separate document classes.

    \textbf{If you are using \href{https://www.overleaf.com/}{Overleaf}, the following steps work in one click.}
    \item Compilation Template:
        \begin{enumerate}
            \item Windows:Double click the `artratex.bat'.
            \item Linux or MacOS: {\scriptsize \verb|terminal| -> \verb|chmod +x ./artratex.sh| -> \verb|./artratex.sh xa|}
            \item Any system: you can use the LaTeX editor to open the Thesis.tex file and select the xelatex compilation engine to compile it.
        \end{enumerate}
    \item Error handling: If you encounter any problems during compilation, please check the "Frequently Asked Questions" (section ~\ref{sec:qa}) first.
\end{enumerate}

\section{Document Catalog Introduction}

\subsection{Thesis.tex}

Thesis.tex is the main document, which designs and plans the overall framework of the thesis, and by reading it you can understand the construction of the entire thesis framework.

\subsection{Compile Scripts}

\begin{itemize}
    \item Windows: Double-click on the Dos script artratex.bat to get a fully compiled PDF document, which exists to help beginners who do not understand the LaTeX compilation process to cross the first hurdle of compilation, please do not distribute and receive this script via e-mail to guard against the potential risks of Dos scripts.
    \item Linux or MacOS: run in terminal
        \begin{itemize}
            \item \verb|. /artratex.sh xa|: get the fully compiled PDF document
            \item \verb|. /artratex.sh x|: fast compile mode
        \end{itemize}
    \item Full compilation means running \verb|xelatex+bibtex+xelatex+xelatex| to properly generate all citation links, such as table of contents, references and citations, etc. If no new references are added during the writing process, you can use fast compilation, i.e., run the LaTeX compilation engine only once to reduce compilation time.
\end{itemize}

\subsection{Tmp folder}
If you use \href{https://www.overleaf.com/}{Overleaf}, you can skip this part.

After running the compilation script, the documents generated by the compilation are stored in the Tmp folder, including the compilation of the PDF document, its existence is to maintain the neatness of the workspace, because a good mood is very important.

\subsection{Style folder}

contains the definition files and configuration files of the NEU-Thesis document classes, which can be modified to achieve specific template settings. If you need to update a template, you can usually just replace the old one with the new style file.

\begin{enumerate}
    \item neuthesis.cls: The document class definition file, through which the core format of the thesis is defined.
    \item neuthesis.cfg: document class configuration file, set e.g. the directory is displayed as "directory" instead of "catalog".
    \item artratex.sty: common macro package and document settings, such as reference style, citation style, header and footer settings. These functions have switch options, and often just need to be enabled in the following command in Thesis.tex, usually without modifying artratex.sty itself.
        
        \path{\usepackage[options]{artratex}} 
    \item artracom.sty: custom commands and recommended locations for adding macro packages.
\end{enumerate}

\subsection{Tex folder}

The folder contains all the entity contents of the thesis, which is normally one of the main locations to focus on and modify when \textbf{writing academic papers using NEU-Thesis, note: all files must be encoded in UTF-8, otherwise garbled text will appear after compilation}, as described in the following detailed categories:

\begin{itemize}
    \item Frontpage.tex: for the English and Chinese cover of the paper and the English and Chinese abstract. \textbf{thesis cover will automatically switch to the corresponding format according to the English degree name such as Bachelor, Master, or Doctor}.
    \item Mainmatter.tex: index the Chapters that need to appear. when you start writing the thesis, you can index only the current chapter for a quick compilation view, and then index all chapters when the thesis is finished.
    \item Chap{\_}xxx.tex: for each chapter of the main body of the paper, which can be added and written as needed.
    \item Appendix.tex: for the content of the appendix
    \item Backmatter.tex: for the published article information and the acknowledgement section, etc.
\end{itemize}

\subsection{Img folder}

This folder is used to place the image files needed in the thesis, and supports the following formats: .jpg, .png, .pdf. Among them, \verb|neu_logo.pdf| is the emblem of Northeastern University. It is not recommended to build subdirectories for each section image, even if there are many images, if the naming rules are reasonable, the image search is also very convenient.

\subsection{Biblio folder}

\begin{enumerate}
    \item ref.bib: Reference information base.
    \item gbt7714-xxx.bst: documentation style definition file conforming to the national standard. Developed by \href{https://github.com/zepinglee/gbt7714-bibtex-style}{zepinglee} and meets the requirements of the latest national standard. For issues related to documentation styles, please consult the documentation provided by the developer and it is recommended to track its updates appropriately.
\end{enumerate}

\section{Common Usage Questions}\label{sec:qa}

\begin{enumerate}
    \item The template has been tested and approved on \href{https://www.overleaf.com/}{Overleaf}. If there is a compilation error after downloading the template, then see \href{https://github.com/sci-m-wang/NEU-Thesis-en/wiki}{NEU-Thesis and LaTeX知识小站}'s \href{https://github.com/sci-m-wang/NEU-Thesis-en/wiki/%E7%BC%96%E8%AF%91%E6%8C%87%E5%8D%97}{guideline}。

    The \item template document is encoded in UTF-8 encoding. All files must be encoded in UTF-8, otherwise the compiled generated document will have garbled text. If you have the problem that the text editor cannot open the document or open the document with garbled text, please check the editor's support for UTF-8 encoding. If you use WinEdt as a text editor (\textbf {not recommended}), you should put in its Options -> Preferences -> wrapping tab two kinds of Wrapping Modes:
        
        TeX;HTML;ANSI;ASCII|DTX...
        
        Modify to: TeX;\textbf{UTF-8|ACP;}HTML;ANSI;ASCII|DTX...
        
        Also, uncheck Enable ANSI Format in Options -> Preferences -> Unicode.

    \item Recommend to choose xelatex or lualatex compilation engine to compile Chinese documents. The default setting for compilation scripts is xelatex compilation engine. You can also choose not to use script compilation, e.g. compile directly with the LaTeX text editor. Note: The default setting for compiling with the LaTeX text editor is the pdflatex compilation engine. If you choose the xelatex or lalatex compilation engine, please go to the drop-down menu and select it. Full compilation is required for proper generation of reference links.

    Introduction to using \item Texmaker
        \begin{enumerate}
            \footnotesize
            \item Use Texmaker to "Open (Open)" Thesis.tex.
            \item menu "Options" -> "Define as Master Document"
            \item menu "Customize (User)" -> "Custom Commands" -> "Edit User Commands " -> select "Command 1" on the left, fill in Auto Build for "Menu Item" on the right -> click "Wizard" below Wizard" -> "Add": xelatex + bibtex + xelatex + xelatex + pdf viewer -> click "Finish (OK) "
            \item uses Auto Build to compile source files with ungenerated reference links, you can use xelatex only to compile source files with correctly generated reference links.
            \item Compilation is complete, "View (View)" PDF, "ctrl+click" in the PDF to link to the corresponding source file.
        \end{enumerate}
    
    \item templates are designed with adaptability in mind where possible. All entries such as acknowledgements are created through the most generic.

        \verb+\chapter{item name}+  and \verb+\section*{item name}+

        This is done explicitly (see Backmatter.tex), so that it can be added, placed, and modified at will, as in the general section. The chart directory name can be changed in neuthesis.cfg.

    \item Set document style: search for keywords in artratex.sty to locate the appropriate command, and then modify
        \begin{enumerate}
            \item Body line spacing: enable and set \verb|\linespread{1.5}|, default 1.5x line spacing.
            \item Reference line spacing: modify \verb|\setlength{\bibsep}{0.0ex}|
            \item Directory display level: modify \verb|\setcounter{tocdepth}{2}|
            \item Document hyperlink colors and their display: modify \verb|\hypersetup|
        \end{enumerate}

    \item The method for switching fonts within a document: \
        \begin{itemize}
            \item 宋体:东北大学论文模板NEU-Thesis or \textrm{东北大学论文模板NEU-Thesis}
            \item Bold Song: {\bfseries Northeastern University Thesis Template NEU-Thesis} or \textbf{Northeastern University Thesis Template NEU-Thesis}
            \item bold: {\sffamily Northeastern University Thesis Template NEU-Thesis} or \textsf{Northeastern University Thesis Template NEU-Thesis}
            \item Bold Bold: {\bfseries\sffamily Northeastern University Thesis Template NEU-Thesis} or \textsf{\bfseries Northeastern University Thesis Template NEU-Thesis}
            \item 仿宋:{\ttfamily 东北大学论文模板NEU-Thesis} or \texttt{NEU-Thesis东北大学论文模板NEU-Thesis}
            \item 粗仿宋:{\bfseries\ttfamily Northeastern University Thesis Template NEU-Thesis} or \texttt{\bfseries Northeastern University Thesis Template NEU-Thesis}
            \item italic: {\itshape Northeastern University Thesis Template NEU-Thesis} or \textit{Northeastern University Thesis Template NEU-Thesis}
            \item Bold italic: {\bfseries\itshape Northeastern University Thesis Template NEU-Thesis} or \textit{\bfseries Northeastern University Thesis Template NEU-Thesis}
        \end{itemize}

    \item The text on the underline of the cover is not centered on the underline because there are headers in front of the underline, causing the text to be centered on the page or centered on the underline. The current cover takes page centering. If you need to adjust the position of the text on the underline, you can use the \verb|\hspace{+/- n.0em}| command to insert or remove n spaces for manual adjustment, such as

        \verb|\advisor{\hspace{+3.0em} xxx~researcher~xxx units}|
                
    Sometimes the underline looks inconsistent in thickness, which is a display problem and prints normally.

\end{enumerate}


